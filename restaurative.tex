\RequirePackage[l2tabu, orthodox]{nag}
\documentclass[french]{beamer}
\input{preamble/packages}
\input{preamble/redac}

%\setbeamertemplate{headline}[singleline]
%\setbeamertemplate{footline}[authortitle]

\title[La justice restaurative pour le traitement des VSS]{La justice restaurative pour le traitement des violences sexistes et sexuelles}
\subtitle{Inclure les auteurs pour libérer la parole}
\author[Jade \and Marin \and Olivier]{Jade Fillaudeau \and Marin Le Core \and Olivier Cailloux}
\institute[Paris-Dauphine]{Université Paris-Dauphine}
\date{\formatdate{31}{1}{2025}}

\begin{document}
\begin{frame}[plain]
	\tikz[remember picture,overlay]{
		\path (current page.south west) node[anchor=south west, inner sep=0] {
			\includegraphics[height=8mm]{Dauphine-Noir.png}
		};
		\path (current page.south east) node[anchor=south east, inner sep=0] {
			\includegraphics[height=1cm]{LAMSADE95.jpg}
		};
		\path (current page.south) ++ (0, 4em) node[anchor=south, inner sep=0] {
			\scriptsize\textcolor{blue}{\url{https://github.com/oliviercailloux/restaurative}}
		};
	}
	\titlepage
\end{frame}
\addtocounter{framenumber}{-1}

\begin{frame}
	\frametitle{\translate{Outline}}
	\tableofcontents[hideallsubsections]
\end{frame}

\AtBeginSection{
	\begin{frame}
		\frametitle{\translate{Outline}}
		\tableofcontents[currentsection, hideothersubsections]
	\end{frame}
}

\section{Les freins à la libération de la parole}
\subsection{Obstacles systémiques}
\begin{frame}
	\frametitle{Obstacles systémiques}
	\begin{itemize}
		\item Poids des tabous sociaux et culturels autour des violences sexistes et sexuelles. → silence pendant des années dans la communauté de Hollow Water
			\begin{quote}
				\textit{"Quand on parle de violences, notamment de violences sexuelles, on parle de traumatismes, on parle de choses que finalement, on est très peu nombreux·ses à ne pas avoir vécu dans nos vies, voir dans nos enfances."} (Transcription 1.json)
			\end{quote}
			\begin{quote}
				\textit{"#MeToo c'est difficile, #MeToo c'est violent, #MeToo c'est dur. Je pense que vous êtes nombreux et nombreuses ici a peut-être avoir ressenti ce que j'ai pu ressentir tout à l'heure en assistant à des conversations, des petites accélérations cardiaques un peu bizarres, des petits malaises, des petites nausées."} (Transcription 1.json)
				\end{quote}
			\begin{quote}
				\textit{"Dans la communauté de Hollow Water, le silence autour des violences sexuelles était profondément enraciné dans les tabous culturels et sociaux. Ce silence a perduré pendant des années, empêchant les victimes de s'exprimer et de chercher de l'aide."} (Hollow.vtt)
			\end{quote}
		\item Peur des représailles ou du jugement social pour les victimes. → jugements au moment de porter plainte pour certaines minorités
			\begin{quote}
				\textit{"Il y a aussi un énorme facteur de peur. Il y a la peur du cyberharcèlement, il y a la peur de la réception, de la réaction de sa famille, de ses proches, de son entourage professionnel, mais aussi la peur des poursuites judiciaires et notamment des poursuites en diffamation."} (Transcription 3.txt)
			\end{quote}
	\end{itemize}
	\begin{block}{Hollow Water}
		Hollow Water est une communauté qui a vécu dans le silence pendant de nombreuses années. Ce silence a finalement été brisé, entraînant des changements et des développements significatifs au sein de la communauté.
	\end{block}
\end{frame}

\subsection{Limites du système judiciaire}
\begin{frame}
	\frametitle{Les limites du système judiciaire classique}
	\begin{itemize}
		\item Un processus centré sur la preuve, où la parole de la victime est souvent mise en doute.
		\item Classements sans suite et taux de condamnation faibles : un facteur de découragement. → principes à respecter pour pouvoir condamner quelqu’un
		\item Absence d’un dialogue direct avec l’agresseur : l’agresseur ne se remet pas en question de la même manière
	\end{itemize}
\end{frame}

\subsection{Conséquences du silence}
\begin{frame}
	\frametitle{Conséquences du silence sur les individus et les communautés}
	\begin{itemize}
		\item Traumatisme prolongé pour les victimes. (cf arbre généalogique)
		\item Transmission intergénérationnelle des violences et maintien des dynamiques dysfonctionnelles.
	\end{itemize}
\end{frame}

\section{La justice restauratrice}
\subsection{Espace de parole pour les victimes}
\begin{frame}
	\frametitle{Un espace de parole pour les victimes}
	\begin{itemize}
		\item Encouragement à s’exprimer dans un cadre sécurisé et non punitif. (cercle de parole de Hollow Water)
		\item Reconnaissance de la souffrance par les agresseurs et la communauté. (groupe de parole → exemple des cercles de parole sur les accidents de la route)
	\end{itemize}
\end{frame}

\subsection{Responsabilisation des auteurs}
\begin{frame}
	\frametitle{Responsabilisation des auteurs : un levier pour transformer les comportements}
	\begin{itemize}
		\item Travail sur le déni : amener les auteurs à reconnaître la gravité de leurs actes. (ex des parents dans le documentaire)
		\item Rôle actif des agresseurs dans le processus de réparation : écouter, s’excuser, et s’engager à changer.
	\end{itemize}
\end{frame}

\subsection{Implication de la communauté}
\begin{frame}
	\frametitle{Implication de la communauté : une réponse collective aux VSS}
	\begin{itemize}
		\item Sensibilisation et responsabilisation collective pour prévenir les violences.
		\item Dynamique d’entraide : exemple de Hollow Water, où la communauté joue un rôle central dans la réhabilitation.
	\end{itemize}
\end{frame}

\section{Impacts et limites}
\subsection{Effets positifs}
\begin{frame}
	\frametitle{Des effets positifs sur les victimes et la communauté}
	\begin{itemize}
		\item Libération progressive de la parole grâce au soutien collectif et à la reconnaissance publique. (augmentation de la prise de parole dans la communauté)
		\item Amélioration du climat de confiance entre victimes et communauté.
	\end{itemize}
\end{frame}

\subsection{Changement durable}
\begin{frame}
	\frametitle{Un modèle qui favorise le changement durable}
	\begin{itemize}
		\item Transformation des dynamiques de silence et de déni dans les communautés.
		\item Réduction des récidives grâce à la responsabilisation des agresseurs.
	\end{itemize}
\end{frame}

\subsection{Défis à relever}
\begin{frame}
	\frametitle{Des défis à relever pour généraliser la justice restaurative}
	\begin{itemize}
		\item Résistances culturelles et institutionnelles : l’attachement au modèle punitif.
		\item Ressources nécessaires : temps, expertise, et engagement collectif.
	\end{itemize}
\end{frame}

\begin{frame}[plain]
	\addtocounter{framenumber}{-1}
	\begin{center}
		\huge
		\textit{Merci pour votre attention !}
	\end{center}
\end{frame}

\end{document}

\appendix
\AtBeginSection{
}

\begin{frame}[allowframebreaks]
	\frametitle{\refname}
 	\bibliography{zotero}
\end{frame}

\clearpage\pdfbookmark{License}{License}
\begin{frame}[plain]
	\frametitle{License}
	This presentation, and the associated \LaTeX{} code, are published under the \href{https://opensource.org/licenses/MIT}{MIT license}. Feel free to reuse (parts of) the presentation, under condition that you cite the author.
	
	Credits are to be given to \hrefblue{https://www.lamsade.dauphine.fr/~ocailloux/}{Olivier Cailloux}, Université Paris-Dauphine.
\end{frame}
\addtocounter{framenumber}{-1}
\end{document}

\begin{frame}
	\frametitle{Title}
	\begin{itemize}
		\item Item
	\end{itemize}
\end{frame}

\begin{frame}
	\frametitle{Title}
	\begin{block}{Block}
%		\setlength\abovedisplayskip{1 ex}% reduce space above equations
		\begin{itemize}
			\item Item
		\end{itemize}
	\end{block}
	\begin{itemize}
		\item Item
	\end{itemize}
\end{frame}

